\documentclass[acmtoc, authorversion]{acmart}

\usepackage{booktabs} % For formal tables

% TOG prefers author-name bib system with square brackets
\citestyle{acmauthoryear}
\setcitestyle{square}


\usepackage[ruled]{algorithm2e} % For algorithms
\renewcommand{\algorithmcfname}{ALGORITHM}
\SetAlFnt{\small}
\SetAlCapFnt{\small}
\SetAlCapNameFnt{\small}
\SetAlCapHSkip{0pt}
\IncMargin{-\parindent}

% Metadata Information
%\acmJournal{TOG}
%\acmVolume{9}
%\acmNumber{4}
%\acmArticle{39}
%\acmYear{2010}
%\acmMonth{3}

% Copyright
%\setcopyright{acmcopyright}
%\setcopyright{acmlicensed}
%\setcopyright{rightsretained}
%\setcopyright{usgov}
\setcopyright{usgovmixed}
%\setcopyright{cagov}
%\setcopyright{cagovmixed}

% DOI
%\acmDOI{0000001.0000001_2}

% Paper history
\received{April 2018}
%\received{March 2009}
%\received[final version]{June 2009}
%\received[accepted]{July 2009}


% Document starts
\begin{document}
% Title portion
\title{A survey on Deprecating the Observer Pattern}

\author{Joel Bartelheimer}
\affiliation{%
	\institution{Technische Hochschule Mittelhessen}
	\city{Giessen}
	\state{Hessen}
	\postcode{23185}
	\country{Germany}}
\email{joel.bartelheimer@mni.thm.de}


\renewcommand\shortauthors{Bartelheimer, J. }

\begin{abstract}
	Multifrequency media access control has been well understood in
	general wireless ad hoc networks, while in wireless sensor networks,
	researchers still focus on single frequency solutions. In wireless
	sensor networks, each device is typically equipped with a single
	radio transceiver and applications adopt much smaller packet sizes
	compared to those in general wireless ad hoc networks. Hence, the
	multifrequency MAC protocols proposed for general wireless ad hoc
	networks are not suitable for wireless sensor network applications,
	which we further demonstrate through our simulation experiments. In
	this article, we propose MMSN, which takes advantage of
	multifrequency availability while, at the same time, takes into
	consideration the restrictions of wireless sensor networks. Through
	extensive experiments, MMSN exhibits the prominent ability to utilize
	parallel transmissions among neighboring nodes. When multiple physical
	frequencies are available, it also achieves increased energy
	efficiency, demonstrating the ability to work against radio
	interference and the tolerance to a wide range of measured time
	synchronization errors.
\end{abstract}


%
% The code below should be generated by the tool at
% http://dl.acm.org/ccs.cfm
% Please copy and paste the code instead of the example below.
%
\begin{CCSXML}
	<ccs2012>
	<concept>
	<concept_id>10011007.10011074.10011075.10011077</concept_id>
	<concept_desc>Software and its engineering~Software design engineering</concept_desc>
	<concept_significance>500</concept_significance>
	</concept>
	<concept>
	<concept_id>10011007.10011074.10011075.10011078</concept_id>
	<concept_desc>Software and its engineering~Software design tradeoffs</concept_desc>
	<concept_significance>300</concept_significance>
	</concept>
	<concept>
	<concept_id>10011007.10011074.10011075.10011079</concept_id>
	<concept_desc>Software and its engineering~Software implementation planning</concept_desc>
	<concept_significance>300</concept_significance>
	</concept>
	</ccs2012>
\end{CCSXML}

\ccsdesc[500]{Software and its engineering~Software design engineering}
\ccsdesc[300]{Software and its engineering~Software design tradeoffs}
\ccsdesc[300]{Software and its engineering~Software implementation planning}
%
% End generated code
%


\keywords{design patter, observer pattern, event handling, data-flow language, reactive programming, user interface programming, scala}



\maketitle

%input

\section{Introduction}

In contrast to traditional batch mode programs modern applications are reactive and event driven. 
examples like the GUI 
reactive is hard and errorprone, dealing with continuous event occourance and user input requires a considerable amount of engineering.
A programming paradigm well suited for these event-driven and interative applications is reactive programming
reactive programming gained popularity

early approach in reactive programming is the use of the observer Pattern

in the paper ``Deprecating the Observer Pattern'' the authors criticize the use of the observer pattern for reactive programs,
the title already is a hard claim by itself.



\subsection{UI Event handling}

\section{Reactive programming}

\subsection{Functional reactive programming (FRP)}

first in functional languages by~\cite{Elliott}

signals behaviour~\cite{Bainomugisha:2013}

\section{Observer Pattern}

\cite{Gamma:1995}

\subsection{Violations of software engineering principles}

\subsubsection{Side-effects}

the observer promotes side effects

\subsubsection{Encapsulation}

\subsubsection{Composability}

\subsubsection{Resource management}

\subsubsection{Separation of concerns}

\subsubsection{Data consistency}

\subsubsection{Uniformity}

\subsubsection{Abstraction}

\subsubsection{Semantic distance}

\section{Scala.React}

\subsection{First class events}

\subsection{Reactor}

\subsection{Embedded dataflow langugage}

\section{Evolution of reactive programming frameworks}

\subsection{Reactive databinding}

Font-end frameworks inspired by the Flapjax reactive language~\cite{Meyerovich:2009}

Metor example: ~\cite{hochhaus2016meteor}

based on fl

\subsection{Reactive streams}

% Bibliography
\bibliographystyle{ACM-Reference-Format}
\bibliography{sources}

\end{document}
